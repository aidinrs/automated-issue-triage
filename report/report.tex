\documentclass[conference]{IEEEtran}
\IEEEoverridecommandlockouts
% The preceding line is only needed to identify funding in the first footnote. If that is unneeded, please comment it out.
\usepackage{cite}
\usepackage{amsmath,amssymb,amsfonts}
\usepackage{algorithmic}
\usepackage{graphicx}
\usepackage{textcomp}
\usepackage{xcolor}
\def\BibTeX{{\rm B\kern-.05em{\sc i\kern-.025em b}\kern-.08em
    T\kern-.1667em\lower.7ex\hbox{E}\kern-.125emX}}
\begin{document}

\title{Labeling Questions Asked On GitHub Issue Trackers\\
% {\footnotesize \textsuperscript{*}Note: Sub-titles are not captured in Xplore and should not be used}
% \thanks{Identify applicable funding agency here. If none, delete this.}
}

\author{\IEEEauthorblockN{Aidin Rasti}
\IEEEauthorblockA{\textit{Electrical Engineering and Computer Science} \\
\textit{University of Ottawa}\\
Ottawa, Canada \\
arast040@uottawa.ca}
}

\maketitle

% \begin{abstract}
% This document is a model and instructions for \LaTeX.
% This and the IEEEtran.cls file define the components of your paper [title, text, heads, etc.]. *CRITICAL: Do Not Use Symbols, Special Characters, Footnotes, 
% or Math in Paper Title or Abstract.
% \end{abstract}

\begin{abstract}
  One of the issues faced by the maintainers of popular open source software is the triage 
  of new reported issues. Many of the issues submitted to issue trackers are questions.
  Many people ask questions on issue trackers about their problem instead of using a proper QA 
  website like StackOverflow. This may seem insignificant but for many of the big projects 
  with millions of users this leads to spamming of the issue tracker. Reading and labelling
  these unrelated issues manually is a serious time consuming task and these unrelated questions 
  add to the burden. In fact, most often maintainers demand to not submit questions in the 
  issue tracker.

  To address this problem, first we have implemented a pattern based method
  to clean text of issues, we removed noise like logs, stack traces, environment variables, 
  error message, etc.
  Second, we have implemented a classification-based approach to automatically label unrelated questions. 
  Empirical evaluations on a dataset of more than 102,000 records shows that our approach 
  can label questions with an accuracy of over 81\%. 
\end{abstract}  

\begin{IEEEkeywords}
  —issue tracking system, natural language processing, machine learning, mining software repositories
\end{IEEEkeywords}

\section{Introduction}
With the prevalence of open source software, authors of projects kindly carry on 
the responsibility of supporting users and providing documentation. The relation 
between users and authors is a two-way relation, developers rely on feedback and reports from 
users to improve their software. These feedback come in the form of bug reports, questions, 
features, suggestions, and enhancements, or what we technically call "Issue"s. Most developers
prefer to use the Issue Tracking software just as a mean for developing the software, it is preferred 
that questions related to the workings or documentation of the software be directed to forums or QA
websites. And by questions we mean everything that is, asking about how to fix an error/problem, 
asking about features and documentation, asking for help, etc.

These unrelated questions only add to the clutter of Issues Trackers, especially in bigger projects.
Developers have to put tremendous amount of effort to triage or close issues. Providing a 
simple automated tool to detect potential unrelated questions can help project managers 
and software developers to focus on more practical issues.

XXX Other works of software defect classification focus on classifying bugs, 
features, enhancements, etc. Most of them don't even include questions. Since most of them are 
based on the data provided from internal issue trackers of projects like Mozilla, jBoss, Apache, they are
trained on a structured and clean dataset. However, for developers that use the built-in issue tracker
of platforms like GitHub, the problem still remains. Moreover, we want to focus on implementing 
a binary classifier to filter questions, instead of the multi-class classifiers that other 
works propose. Maybe we can say that the goals are a little different or even complementary. 
For example, we can first filter spams by using our proposed classifier and then when 
we are more confident that the reported issue is related, we can use another classifier to 
automatically categorize it. This may improve the accuracy of automatic categorization too. Also, this 
enables deployment of those classifiers on a broader range of projects.

In This work we want to investigate the feasibility of detecting unrelated questions
in issue tracker of popular open source software on platforms like GitHub. We want to evaluate the performance
of different classification algorithms. And finally, to compare the performance of a binary classification 
(question, not-question) approach to other multi-class methods used in other works.

\begin{figure}[t]
  \centerline{\includegraphics[width=3.5in]{./figures/q-issue.png}}
  \caption{A question submitted to the issue tracker of rails project on GitHub.}
  \label{fig:qissue}
\end{figure}

\begin{figure*}[t]
  \centerline{\includegraphics{./figures/process.png}}
  \caption{Summary of our approach at a high level.}
  \label{fig:process}
\end{figure*}

As mentioned earlier, our goal is to train a binary classifier to automatically label questions.
We have used previously available dataset of GitHub issues provided by \cite{8816794}.
Cleaning and pre-processing the dataset narrowed down the count of our available 
records to 102,198. The pre-processing part was the hardest part of our implementation
since text of issues usually contain machine generated texts like logs and stack traces. 
After extracting the text part of issues, we have used state of the art sentence embedding techniques
to convert text data to numerical vectors. By evaluating several classification 
algorithms, we achieved the best result with the XXX SVM algorithm.



First, In Section \ref{relworks} we discuss other related works. 
In Section \ref{approach} we describe our dataset, data pre-precessing methods, 
and the technical implementation. In Section \ref{eval} we show the performance of our implementation
and discuss the results of different classification methods.
Finally, Section \ref{conclusion} concludes this report.


\section{Related Works}\label{relworks}

Work of Antoniol et al. \cite{10.1145/1463788.1463819} is one the defect classification works. They trained  
classifiers with Naive Bayes, ADTree, and Logistic Regression on the text data of 1,800 issues extracted 
from Mozilla, Eclipse, and JBoss. Their work is focused on classifying whether a submitted bug is bug or 
non-bug (enhancement, feature request, etc). Zhou et al. \cite{6976097} improve on work 
of \cite{10.1145/1463788.1463819} by including other structures attributes available in a issue tracker,
such as assignee, dates, severity, into the machine learning model. Pingclasai et al. \cite{6754344} build a 
classifier using LDA topic modeling for reclassification of issues. 
They used decision tree, naive Bayes classifier, and logistic regression to classify issues to bug and not-bug. 
Another work by Cubranic and Murphy \cite{1814517} use Naive Bayes for text classification 
to find an assignee for a bug report. Hanmin and Xin use a LSTM neural network \cite{10.1145/3275219.3275239}
to classify issues to bug and non-bugs, they showed that with a LSTM neural network they can achieve better 
performance than some other machine learning based approaches.

Kochhar et al. \cite{6923127} address the problem of misclassified reports in issue trackers. They have 
extracted features from 7,000 issues from five open source projects to classify them into 13 categories. Unlike
the other mentioned works, Kochhar et al. build a multi-class classifier using textual and structural data from 
issues. But they categories does not include questions.

To the best of our knowledge we couldn't find any issue classification works which included questions. Furthermore, 
almost all of the mentioned works were done on a small dataset (less than 8,000 records) 
collected from internal issue trackers.


\section{Approach}\label{approach}
In summary, our approach involves three main phases:
\begin{enumerate}
  \item The first phase is cleaning and pre-processing the text data of issues. 
  We used around 400 regular expressions to remove noise from the text of issues. 
  Most of the issues are not properly formatted and they contain stack traces, log lines, code snippets, 
  command lines, environment variables, configurations, ip addresses, identifiers, flags, timestamps, etc.
  We have also applied the standard NLP pre-processing tasks such as tokenization and removing punctuations.
  \item The second phase is computing a sentence embedding of each text document provided by 
  the previous step. We have used the Sentence Bert \cite{reimers-2019-sentence-bert} technique to convert our
  documents to a high dimensional vector representation. The Sentence Bert is one the 
  best performing sentence embedding techniques according to benchmarks \cite{reimers-2019-sentence-bert}.
  \item The final phase is training a classifier on the extracted features. We have evaluated
  the performance of SVM, Decision Tree (C4.5), Logistic Regression, Random Forest and k-NN 
  classification algorithms.
\end{enumerate}

In the following sections we explain our dataset, pre-processing methods, a few background definitions, and
our implementation.

\subsection{Dataset}
We used the RapidRelease \cite{8816794} dataset. It contains 2 million issues from active and popular 
GitHub repositories of 18 programming languages. The provided dataset is a SQL database.
The label of each issue, which are applied by the project maintainers, is also available in this dataset.

With a simple query we extracted frequently applied labels. you 
can see the count of the top 10 labels in Table \ref{tab:labels}. 
We have selected these top labels, \verb|bug|, \verb|duplicate|, \verb|enhancement|, \verb|wontfix|,
\verb|feature|, \verb|improvement|, and \verb|question| to find more labels based on them.
For example, some projects may use the label "type: bug" to represent a bug. Also, many of the issues 
have more than one label. The labels are stored in the database as string in a comma separated format. 
For example, an issue with both labels of bug and duplicate, is stored as ``bug,duplicate". 
For each of these base labels, we queried issues that contained their strings. For example,
for features we looked for labels that contained the string ``feature". We also queried their usage count and
extracted top top applied labels for each category.
For example, The top five applied labels that contained the ``feature" string 
are: ``feature", ``feature request", ``kind/feature", ``cat:feature", ``type: feature".
A label with a higher usage indicates that it is a relevant label for the intended purpose. This query resulted
in thousands of records for each of the base labels, however, many of them were only applied
a handful of times. Therefore we have only selected labels from these lists that 
at least were used 50 times. For our classifier we have two categories of \verb|not-question| 
and \verb|question|. We used the base labels \verb|bug|, \verb|duplicate|, \verb|enhancement|, \verb|wontfix|,
\verb|feature|, \verb|improvement|, and their similar labels for the \verb|not-question| category. We used
the \verb|question| or similar labels for the \verb|question| category. Given that many issues have 
more than one label, we removed issues that had labels like ``bug,question'' or ``support,bug'' from 
the list of frequent labels. We used this list of labels to extract issues. The list of labels
used can be found in our code repository in the \verb|labels| folder for each one of the base labels.
With this process we extracted XX 46,549 issues for the \verb|question| category and XX 280,829 for 
the \verb|not-question|.

\begin{table}[t]
  \caption{Top 10 Applied Labels}
  \begin{center}
  \begin{tabular}{|c|c|}
  % \hline
  % \multicolumn{2}{|c|}{\textbf{Top 10 Applied Labels}} \\
  \hline
  % \cline{2-4} 
  \textbf{Label} & \textbf{Count}\\
  \hline
  bug           & 65735 \\
  \hline
  enhancement   & 43673 \\
  \hline
  question      & 34248 \\
  \hline
  cla: yes      & 17476 \\
  \hline
  duplicate     & 12210 \\
  \hline
  feature       & 9235 \\
  \hline
  stale         & 7996 \\
  \hline
  documentation & 7814 \\
  \hline
  invalid       & 7658 \\
  \hline
  cla signed    & 7306 \\
  \hline
  % \multicolumn{4}{l}{$^{\mathrm{a}}$Sample of a Table footnote.}
  \end{tabular}
  \end{center}
  \label{tab:labels}
\end{table}

\subsection{Date cleaning and pre-processing}
After extracting issues the pre-processing phase begins. GitHub issues are written in the Markdown format.
We used the \verb|marked| \cite{web:marked} library to parse and compile issues, this library compiles
Markdown to HTML. Following compiling issues to HTML we extracted the text part of issues from HTML.
Also, during the translation to HTML we removed image, table, code, pre, and header tags. If the
author of an issue had used the Markdown properly then after removing these tags only the text part
of an issues would have remained. However, many developers do not submit issues in the proper format and we
still have to further clean texts. 

We applied more than 400 regular expressions to remove the remaining noise from text of issues. In 
particular we have removed dependency trees, emojis, code snippets, comments, logs,
error message, stack traces, timestamp, date times, command lines, environment variables, identifiers,
HTML, markup tags, module versions, IP addresses, emails, GitHub user handles, URIs, file paths, and digits.
These regular expressions were designed by manually skimming through texts. Suppose we want to extract
stack traces, most of the stack traces contain several consecutive lines that begin with 
the ``at'' or ``in'' word or they contain the file extension of that specific 
programming language (``.js'', ``.py''). Also, stack traces of a platform like Java or .NET are 
very similar to each other and creating a pattern to extract them is easy.
By manually analyzing our regular expression for stack traces we identified thousands
of occurrences. As another example, suppose we want to remove log lines. Many developers copy logs or debug
output of the software to help diagnose the problem. Almost all of the logs contain a timestamp on
each line, therefore, if we see several consecutive lines that contain timestamps, there is 
a high probability that they are software generated log outputs.

By manually applying our regular expressions and inspecting dozens of results we have fine tuned 
them to not catch incorrect patterns. Of course these regular expressions are not perfect and there are 
still a handful of incorrect matches in a large dataset, but given that they can match hundreds of correct patterns, 
we used them to remove noise from text of issues. We have also removed several punctuations except 
those that indicated end of a sentence (such as .,;!?). The whole list of used regular expression is available
in the code repository.

The next step of pre-processing was filtering issues which were not in English language. We used 
the \verb|langdetect| \cite{web:langdetect} library to determine the language of each issue 
and then filter the ones that were not English.

After cleaning the dataset we concated title and body of issues then we used the nltk library for 
tokenization. We used its default tokenizer which is \verb|TreebankWordTokenizer|. Following 
tokenization we marked issues that had more than five and lower than 200 tokens for training. 
This reduced the records of the \verb|question| category 
to 42,198 and the \verb|not-question| category to 262,425. Also, for the \verb|question| category we only selected
issues that were closed. 


\subsection{Sentence Embeddings}
To understand the concept of sentence embeddings we should first understand word embeddings.
word embeddings are essentially vector representation of words. There are a few techniques to train
numerical vector representation of words. Word2vec \cite{mikolov2013efficient,mikolov2013distributed} 
and GloVe \cite{pennington2014glove} are the most famous word embeddings algorithms. The advantage of
word embeddings over older techniques like One-hot encoding and Bag-of-words is that semantically similar
words appear close to each other in the vector space and the relation among the words are preserved.
One popular example of this case is the similarity between the words ``king'' and ``queen'', for example
we can calculate a vector close to the vector of word ``queen'' by calculating the formula:
\begin{equation}
  king-men+woman=queen
\end{equation}
XX picture of glove capital country relation pic
Word embeddings give us a numerical vector for each word. Sentence embeddings are an extension to 
the word embeddings technique. Sentence embeddings are vector representation of a sentence, paragraph 
or even a whole text document. One naive way of calculating the sentence embeddings of a text or sentence 
is to average its word embeddings, however, this method does not perform well for NLP tasks. 
Sentence-BERT \cite{reimers-2019-sentence-bert} is an state of the art 
algorithm to derive sentence embeddings. Sentence-BERT builds on top of the BERT \cite{devlin2019bert} networks 
to calculate meaningful sentence embeddings. Another algorithm to calculate sentence embeddings is 
Universal Sentence Encoder \cite{cer2018universal}. Vectors generated by both of the two mentioned sentence
embedding algorithms can be compared using the cosine similarity function. Sentence embeddings perform
extremely well for similarity and classification tasks. Since our task is classification, we decided to 
use these sentence embeddings to compute a numerical vector for the text of each issue in our 
cleaned dataset. We computed embeddings using both Universal Sentence Encoder and Sentence-BERT then we trained
classifiers for both of them to compare results. The input to these models is string and the output is 
a numerical vector. We used the \verb|roberta-large-nli-stsb-mean-tokens| pre-trained 
model for Sentence-BERT which is trained over the SNLI \cite{snli:emnlp2015} and Multi NLI \cite{N18-1101} datasets. 
This model generates 1024 dimension vectors. We used the 
\verb|universal-sentence-encoder-large-v5| pre-trained model of Universal Sentence Encoder. This model generates 
512 dimension vectors. This model is trained over a variety of unsupervised data sources from web. They used data
from Wikipedia and question-answer websites. The resulting labeled vectors of both of these embeddings 
are available in our code repository.

\subsection{Training classifiers}
After extracting feature vectors we now have a dataset ready to train classification models. We have two datasets,
one generated from Universal Sentence Encoder and one generated from Sentence-BERT. The Universal Sentence Encoder
records have 512 dimensions while the Sentence-BERT ones have 1024 dimensions. We used the Weka UI toolkit 
\cite{10.1145/1656274.1656278} to train classifiers. We have trained classifiers using the Logistic Regression, 
Decision Tree (C4.5), Random Forest, and Support Vector Machine. In this section we briefly review each of the 
classification algorithms.
\begin{itemize}
  \item\textbf{Logistic Regression:}: Logistic Regression can be used for binary classification. Instead of 
  predicting a continues value like Linear Regression, it predicts the probability belonging to two different 
  classes. During the training process it finds a logistic function to calculate this probability.
  \item\textbf{Decision Tree (C4.5):} The C4.5 decision tree algorithm calculates the information gain ratio 
  for all of the attributes then splits data at each node by the attribute with the highest normalized 
  information gain ratio. It recursively continues to split data until the tree is complete.
  \item\textbf{Random Forest:} Random Forest is an Ensemble Learning algorithm. Basically, it consists of several 
  uncorrelated decision trees, each used for classification. The class with majority of votes becomes 
  the final results of a Random Forest classifier.
  \item\textbf{Support Vector Machine:} SVM is one of the best performing linear classification algorithms. It can 
  classify data by finding the best hyperplane that separates the dataset in two given classes. To allow 
  room for generalization it also selects the hyperplane which has the most margin from both categories. It can 
  also work on non-linearly separable dataset by using a kernel function.
  \item\textbf{k-Nearest Neighbors:} To use the k-NN algorithm for classification the raw dataset becomes the model,
  there is no training process. When a record is queried for its class, the algorithm finds \textit{k} closest 
  data points in the raw dataset according to a distance function (for e.g. Euclidean distance). The class of 
  the queried record is determined by the majority vote of its neighbors. The number of neighbor 
  points (\textit{k}) to use for classification is an input parameter, the best \textit{k} value can be 
  determined by evaluating results over the training split.
\end{itemize}

\section{Empirical Evaluation}\label{eval}
We have implemented our approach to answer our three research questions. Our questions are:
\begin{itemize}
  \item{RQ1)} \textit{Is it possible to prevent spamming in issue trackers of popular open source software? 
  is it possible to label questions asked in issue trackers?}
  \item{RQ2)}\textit{Which classification algorithms yield goods result for this specific task?}
  \item{RQ3)} \textit{Is the binary classification (question, not-question) a good approach? 
  does it help or improve results?}
\end{itemize}

\subsection{RQ1 and RQ2}
In order to answer our first two questions we have evaluated the performance of several popular
classification algorithms over both datasets. One dataset containing feature vectors generated by 
Sentence-BERT algorithm and another generated by the Universal Sentence Encoder algorithm. We have two labels for
classification. The \verb|question| category has XX 42,000 issues. For the \verb|not-question| category 
we randomly chose 60,000 issues from a total XX 282,000. This kept our dataset more balanced while 
not becoming imbalanced. In total our dataset contains XXX 102,000 records. The exact same issues were used 
for both of the sentence embedding algorithms, however, the generated feature vectors have different dimensions, 
vectors of the Sentence-BERT algorithm has 1024 dimension while vectors of the Universal Sentence Encoder has 
512 dimensions. Vectors generated by both these algorithms are normalized and we did not perform any normalization
on them for any of the classification algorithms. We used the Weka toolkit to train classification models. 
If Weka toolkit normalized values by default for any of the classification algorithms, 
we turned it off before training models. 

We trained classifiers with the Logistic Regression, Decision Tree (C4.5), Random Forest, and Support Vector Machine
algorithms. For all of the classifier we ran the training process with its default parameters. You can find 
these parameters in the code repository under the \verb|weka| folder. We trained all of the algorithms over 70\%
of the dataset and we evaluated them over the 30\% test split. Also, we trained each of these algorithms two 
times since we had two datasets. Due to the limited computation resources and time, we were not able to 
perform cross validation. Still a 30\% test split over a dataset with more than 100,000 records is large 
and diverse enough to give reliable results.
The output of Weka for each algorithm is available in our code repository too. 

In Table \ref{tab:results} you can find the accuracy of each classifier. Logistic Regression has the 
highest accuracy among Universal Sentence Encoder data and SVM has the highest accuracy among 
Sentence-BERT. The best classifier is the Logistic Regression trained over Universal Sentence Encoder 
embeddings with an accuracy of 81.68\%. It almost performs two percent better than the best classifier 
trained over the Sentence-BERT embeddings. In addition, we can observe that classifiers trained over 
the Universal Sentence Encoder embeddings have more consistent results. All of its classifiers  
perform good with an accuracy of at least 78\% (except the decision tree). 

\begin{table}[t]
  \caption{Accuracy of classification algorithms trained over both embeddings with a 70-30 split}
  \begin{center}
  \begin{tabular}{|l|c|c|}
  % \hline
  % \multicolumn{2}{|c|}{\textbf{Top 10 Applied Labels}} \\
  \hline
  % \cline{2-4} 
  \textbf{Sentence Embedding} & \textbf{Classifier}&\textbf{Accuracy}\\
  \hline
                            &k-NN                & 78.38\% \\
                            &Decision Tree(C4.5) & 68.12\% \\
  Universal Sentence Encoder&\textbf{Logistic Regression} & \textbf{81.68\%} \\
  (512 features)            &Random Forest       & 79.05\% \\
                            &SVM                 & 78.18\% \\
  \hline
               &k-NN                & 69.88\% \\
               &Decision Tree(C4.5) & 60.41\% \\
  Sentence-BERT&Logistic Regression & 78.42\% \\
  (1024 features)&Random Forest     & 71.35\% \\
               &\textbf{SVM}        & \textbf{79.86\%}\\
\hline
  % \multicolumn{4}{l}{$^{\mathrm{a}}$Sample of a Table footnote.}
  \end{tabular}
  \end{center}
  \label{tab:results}
\end{table}



\begin{table}[h!b]
  \caption{Precision and recall rates of }
  \begin{center}
  \begin{tabular}{|l|c|c|c|c|}
  \hline
  \textbf{Classifier}&\textbf{Class}&\textbf{Precision}&\textbf{Recall}&\textbf{F-Measure}\\
  \hline
  \multicolumn{5}{|c|}{\textit{Universal Sentence Encoder}}\\
  \hline
  k-NN               &NQ&79.2\%&85.8\%&82.3\%\\
                     &Q &77.0\%&67.9\%&72.1\%\\
  \hline
  Decision Tree(C4.5)&NQ&73.0\%&72.6\%&72.8\%\\
                     &Q &61.3\%&61.8\%&61.5\%\\
  \hline
  Logistic Regression&NQ&83.0\%&86.5\%&84.7\%\\
                     &Q &79.6\%&74.8\%&77.1\%\\
  \hline
  Random Forest      &NQ&77.3\%&91.2\%&83.6\%\\
                     &Q &83.1\%&61.8\%&70.9\%\\
  \hline
  SVM                &NQ&76.2\%&91.4\%&83.1\%\\
                     &Q &82.9\%&59.4\%&69.2\%\\
  \hline
  \multicolumn{5}{|c|}{\textit{Sentence-BERT}}\\
  \hline
  k-NN               &NQ&78.0\%&68.3\%&72.8\%\\
                     &Q &61.2\%&72.2\%&66.2\%\\
  \hline                     
  Decision Tree(C4.5)&NQ&66.7\%&66.0\%&66.3\%\\
                     &Q &51.6\%&52.3\%&51.9\%\\
  \hline                   
  Logistic Regression&NQ&80.1\%&84.4\%&82.2\%\\
                     &Q &75.6\%&69.8\%&72.6\%\\ 
  \hline
  Random Forest      &NQ&71.1\%&86.9\%&78.2\%\\ 
                     &Q &72.1\%&48.9\%&58.3\%\\
  \hline                   
  SVM                &NQ&80.6\%&86.8\%&83.6\%\\
                     &Q &78.6\%&69.8\%&73.9\%\\ 
  \hline
  \end{tabular}
  \end{center}
  \label{tab:results2}
\end{table}
To evaluate our approach better we examine other metrics too, accuracy alone is not a good metric for 
evaluation. We have gathered the confusion matrix of each classifier in Fig.~\ref{fig:cfheatmap}. 
By looking at it we see that the hardest part is labeling actual questions, all of the algorithms almost 
struggle at classifying them. Logistic Regression over Universal Sentence Encoder embeddings, which had 
the highest accuracy, has the highest TP rate for the question category too (74.8\%). SVM over 
Universal Sentence Encoder embeddings has the highest true negatives category. 
But The SVM classifier also has the lowest false positive numbers, the Logistic Regression has almost 1000 more 
false positives. In classifiers trained over the Sentence-BERT embeddings, SVM has both the highest TP (69.8\%) 
and the lowest FP rate (13.2\%). But the Random Forest classifier has the highest true negatives with a 
little margin higher than the SVM.

To further compare results, we have provided Precision, Recall, and F-measure rates in Table \ref{tab:results2}.
The Logistic algorithm trained over Universal Sentence Encoder embeddings has the highest F-measure rates.
In the Sentence-BERT embeddings category, the SVM classifier has the highest F-measure. By analyzing the precision 
rates we can still see that Logistic Regression trained with universal Sentence Encoder embeddings still has 
the best performance, its average precision is higher than any other classifier. 
Although, as we can see in Fig.~\ref{fig:cfheatmap} it has much higher false positives than the SVM and 
Random Forest. It may be more sensible to use these two instead of the Logistic Classifier.

\begin{figure*}[t]
  \centerline{\includegraphics[width=7.16in]{./figures/cf-heatmaps8.png}}
  \caption{Heat map of confusion matrix of trained models over both sentence embeddings.}
  \label{fig:cfheatmap}
\end{figure*}

Regarding RQ1, given the evaluations and empirical evidence, we can conclude that it is feasible to classify 
questions in issue trackers of platforms like GitHub with a pretty good precision. We saw that we can achieve 
an accuracy rate of 81.6\%. To answer RQ2 we trained several models with popular algorithms. We assessed two
different Sentence Embedding techniques to find which is more suitable for our task. We observed that the 
Universal Sentence Encoder has better and more consistent performance for our task. The answer to RQ2 is not a 
single algorithm, if we only measure the accuracy metric, the Logistic Regression works best. However, if we
analyze other metrics we mau choose other algorithms. For example, the Random Forest model trained with
Universal Sentence Encoder embeddings has an accuracy of 79.00\% but it has much lower false positives than the 
Logistic Regression classifier.

We noticed that the bottleneck in the accuracy of almost all of the classifiers is categorizing issues that are 
actually questions. They can label issues that are not questions with good accuracy. This problem can be attributed 
to the fact that many of the submitted questions in issues trackers will be converted to other type of issues by 
the maintainers. For example, a submitted question about a feature may interest developers and they decide to 
implement it, thus they will apply a ``feature'' label to the issue which is asking a question. 
The same happens with other labels such as ``bug'' too, a submitted question about an error 
may in fact turn out to be a bug. Since our dataset is not manually labeled for the purpose of our task and we 
depend on the labels provided by authors of projects this problem can't be addressed trivially. 
Manually labeling issues may help address this problem bu it is time consuming. Another approach may be to analyze 
the dataset using an unsupervised algorithm.

\subsection{RQ3}
With RQ3 we are interested in the comparing our approach with other works. However, a direct comparison with other 
works is not possible. As we saw in Section \ref{relworks}, almost all of the works in the defect classification area 
are multi-class classification problems. XX By comparing raw metrics we can reckon that our approach has the highest 
accuracy for classifying questions. Nevertheless, we think that our approach and other works are complementary.
We can apply both of them in a pipeline fashion. First, we can separately label questions submitted to a 
issue tracker, then apply another categorization methods to predict the class of issue.

\subsection{Limitations and Further Improvements}
\label{limits}
Due to the size of our dataset and limited computation power, we were not able tune hyper parameters of 
classification algorithms used in Weka. We used the default parameters set by Weka. 
However, when we searched for the best value for a few of the parameters over a small sample dataset,
we saw a modest improvement in results.

Our initial solution to solve this problem was to train an LSTM network with Word Embeddings. Using lower level 
embeddings with a neural network may improve results. But we don't know how much improvement it can yield.
As explained in Section \ref{approach} after pre-processing the RapidRelease dataset, we only used issues with 
more than five and less than 200 tokens to train models. Also, vectors generated by Word Embedding algorithms 
can have at least 100 dimensions. With this high amount of input and training parameters, training a neural network 
failed on our device. 

As mention above, we filtered issues with long texts, we have only trained models over issue that had lower than 200 
tokens. We removed longer issues because the performance of sentence embedding algorithms degrades 
with longer texts. Still 200 tokens is long enough for most of issues and we also counted tokens after 
removing all the noise (log lines, stack traces) from the text of issue.

\subsection{Threats to Validity}
Like all evaluations our tests are also prone to external validity and generalization. We have only used data from 
the GitHub issue tracker. But we think that the amount of questions asked on issue tracker of projects that don't 
use GitHub (for example, ASF software have their own issue trackers) is very small, since GitHub is the only major 
platform for hosting open source projects. Furthermore, the RapidRelease dataset is diverse and includes issues 
form several developer communities and programming languages. 

Another factor is that we relied on the labels applied by developers of projects. The labels may not completely 
reflect our intended two categories, especially for questions. As we discussed in Subsection \ref{limits} this 
may be the reason why classifiers had less precision for the question category. Overall, we tried to use issues 
that had labels 

\section{Lessons Learned}
The most important lesson I learned was that to not rely on a single solution. It is always best to evaluate several 
approaches to find the best possible fit. Another thing I noticed is that accuracy metric alone is not enough 
to gauge performance. We have to look at other metrics too and analyze more details. The accuracy metric can be 
misleading.

\section{Conclusion}\label{conclusion}
Our goal was analyzing feasibility of detecting unrelated questions submitted to issue trackers in platforms like 
GitHub. To address this problem We used a previously provided dataset of GitHub issues by the RapidRelease work. 
We filtered and pre-processed the raw dataset and narrowed count of issues for analysis to approximately 
102,000 issues. We were able to filter noise, such as stack traces, logs, etc, from text of issues. 
To prepare our textual data for machine learning algorithms, we used two of state-of-the-art sentence embeddings
techniques, Sentence-BERT and Universal Sentence Encoder, to extract numerical feature vectors. Finally, we 
trained classifiers over our dataset with five famous classification algorithms. The best result was yielded
with the Logistic Regression algorithm trained over Universal Sentence Encoder embeddings with the accuracy
rate of 81.68\%. Overall, we conclude that it is possible to label questions submitted to an public issues tracker
with a good probability, however, the 


\bibliographystyle{IEEEtran}
\bibliography{IEEEabrv,references}
\vspace{12pt}
\color{red}

\end{document}

% pre process pseudo code
% transfer learning
% make db dump



























% \section{Ease of Use}

% \subsection{Maintaining the Integrity of the Specifications}

% The IEEEtran class file is used to format your paper and style the text. All margins, 
% column widths, line spaces, and text fonts are prescribed; please do not 
% alter them. You may note peculiarities. For example, the head margin
% measures proportionately more than is customary. This measurement 
% and others are deliberate, using specifications that anticipate your paper 
% as one part of the entire proceedings, and not as an independent document. 
% Please do not revise any of the current designations.

% \section{Prepare Your Paper Before Styling}
% Before you begin to format your paper, first write and save the content as a 
% separate text file. Complete all content and organizational editing before 
% formatting. Please note sections \ref{AA}--\ref{SCM} below for more information on 
% proofreading, spelling and grammar.

% Keep your text and graphic files separate until after the text has been 
% formatted and styled. Do not number text heads---{\LaTeX} will do that 
% for you.

% \subsection{Abbreviations and Acronyms}\label{AA}
% Define abbreviations and acronyms the first time they are used in the text, 
% even after they have been defined in the abstract. Abbreviations such as 
% IEEE, SI, MKS, CGS, ac, dc, and rms do not have to be defined. Do not use 
% abbreviations in the title or heads unless they are unavoidable.

% \subsection{Units}
% \begin{itemize}
% \item Use either SI (MKS) or CGS as primary units. (SI units are encouraged.) English units may be used as secondary units (in parentheses). An exception would be the use of English units as identifiers in trade, such as ``3.5-inch disk drive''.
% \item Avoid combining SI and CGS units, such as current in amperes and magnetic field in oersteds. This often leads to confusion because equations do not balance dimensionally. If you must use mixed units, clearly state the units for each quantity that you use in an equation.
% \item Do not mix complete spellings and abbreviations of units: ``Wb/m\textsuperscript{2}'' or ``webers per square meter'', not ``webers/m\textsuperscript{2}''. Spell out units when they appear in text: ``. . . a few henries'', not ``. . . a few H''.
% \item Use a zero before decimal points: ``0.25'', not ``.25''. Use ``cm\textsuperscript{3}'', not ``cc''.)
% \end{itemize}

% \subsection{Equations}
% Number equations consecutively. To make your 
% equations more compact, you may use the solidus (~/~), the exp function, or 
% appropriate exponents. Italicize Roman symbols for quantities and variables, 
% but not Greek symbols. Use a long dash rather than a hyphen for a minus 
% sign. Punctuate equations with commas or periods when they are part of a 
% sentence, as in:
% \begin{equation}
% a+b=\gamma\label{eq}
% \end{equation}

% Be sure that the 
% symbols in your equation have been defined before or immediately following 
% the equation. Use ``\eqref{eq}'', not ``Eq.~\eqref{eq}'' or ``equation \eqref{eq}'', except at 
% the beginning of a sentence: ``Equation \eqref{eq} is . . .''

% \subsection{\LaTeX-Specific Advice}

% Please use ``soft'' (e.g., \verb|\eqref{Eq}|) cross references instead
% of ``hard'' references (e.g., \verb|(1)|). That will make it possible
% to combine sections, add equations, or change the order of figures or
% citations without having to go through the file line by line.

% Please don't use the \verb|{eqnarray}| equation environment. Use
% \verb|{align}| or \verb|{IEEEeqnarray}| instead. The \verb|{eqnarray}|
% environment leaves unsightly spaces around relation symbols.

% Please note that the \verb|{subequations}| environment in {\LaTeX}
% will increment the main equation counter even when there are no
% equation numbers displayed. If you forget that, you might write an
% article in which the equation numbers skip from (17) to (20), causing
% the copy editors to wonder if you've discovered a new method of
% counting.

% {\BibTeX} does not work by magic. It doesn't get the bibliographic
% data from thin air but from .bib files. If you use {\BibTeX} to produce a
% bibliography you must send the .bib files. 

% {\LaTeX} can't read your mind. If you assign the same label to a
% subsubsection and a table, you might find that Table I has been cross
% referenced as Table IV-B3. 

% {\LaTeX} does not have precognitive abilities. If you put a
% \verb|\label| command before the command that updates the counter it's
% supposed to be using, the label will pick up the last counter to be
% cross referenced instead. In particular, a \verb|\label| command
% should not go before the caption of a figure or a table.

% Do not use \verb|\nonumber| inside the \verb|{array}| environment. It
% will not stop equation numbers inside \verb|{array}| (there won't be
% any anyway) and it might stop a wanted equation number in the
% surrounding equation.

% \subsection{Some Common Mistakes}\label{SCM}
% \begin{itemize}
% \item The word ``data'' is plural, not singular.
% \item The subscript for the permeability of vacuum $\mu_{0}$, and other common scientific constants, is zero with subscript formatting, not a lowercase letter ``o''.
% \item In American English, commas, semicolons, periods, question and exclamation marks are located within
%  quotation marks only when a complete thought or name is cited, such as a title or full quotation. When 
%  quotation marks are used, instead of a bold or italic typeface,
%  to highlight a word or phrase, punctuation should appear outside of the quotation marks. 
%  A parenthetical phrase or statement at the end of a sentence is punctuated outside of the
%   closing parenthesis (like this). (A parenthetical sentence is punctuated within the parentheses.)
% \item A graph within a graph is an ``inset'', not an ``insert''. The word alternatively is preferred to the word ``alternately'' (unless you really mean something that alternates).
% \item Do not use the word ``essentially'' to mean ``approximately'' or ``effectively''.
% \item In your paper title, if the words ``that uses'' can accurately replace the word ``using'', capitalize the ``u''; if not, keep using lower-cased.
% \item Be aware of the different meanings of the homophones ``affect'' and ``effect'', ``complement'' and ``compliment'', ``discreet'' and ``discrete'', ``principal'' and ``principle''.
% \item Do not confuse ``imply'' and ``infer''.
% \item The prefix ``non'' is not a word; it should be joined to the word it modifies, usually without a hyphen.
% \item There is no period after the ``et'' in the Latin abbreviation ``et al.''.
% \item The abbreviation ``i.e.'' means ``that is'', and the abbreviation ``e.g.'' means ``for example''.
% \end{itemize}
% An excellent style manual for science writers is \cite{b7}.

% \subsection{Authors and Affiliations}
% \textbf{The class file is designed for, but not limited to, six authors.} A 
% minimum of one author is required for all conference articles. Author names 
% should be listed starting from left to right and then moving down to the 
% next line. This is the author sequence that will be used in future citations 
% and by indexing services. Names should not be listed in columns nor group by 
% affiliation. Please keep your affiliations as succinct as possible (for 
% example, do not differentiate among departments of the same organization).

% \subsection{Identify the Headings}
% Headings, or heads, are organizational devices that guide the reader through 
% your paper. There are two types: component heads and text heads.

% Component heads identify the different components of your paper and are not 
% topically subordinate to each other. Examples include Acknowledgments and 
% References and, for these, the correct style to use is ``Heading 5''. Use 
% ``figure caption'' for your Figure captions, and ``table head'' for your 
% table title. Run-in heads, such as ``Abstract'', will require you to apply a 
% style (in this case, italic) in addition to the style provided by the drop 
% down menu to differentiate the head from the text.

% Text heads organize the topics on a relational, hierarchical basis. For 
% example, the paper title is the primary text head because all subsequent 
% material relates and elaborates on this one topic. If there are two or more 
% sub-topics, the next level head (uppercase Roman numerals) should be used 
% and, conversely, if there are not at least two sub-topics, then no subheads 
% should be introduced.

% \subsection{Figures and Tables}
% \paragraph{Positioning Figures and Tables} Place figures and tables at the top and 
% bottom of columns. Avoid placing them in the middle of columns. Large 
% figures and tables may span across both columns. Figure captions should be 
% below the figures; table heads should appear above the tables. Insert 
% figures and tables after they are cited in the text. Use the abbreviation 
% ``Fig.~\ref{fig}'', even at the beginning of a sentence.

% \begin{table}[htbp]
% \caption{Table Type Styles}
% \begin{center}
% \begin{tabular}{|c|c|c|c|}
% \hline
% \textbf{Table}&\multicolumn{3}{|c|}{\textbf{Table Column Head}} \\
% \cline{2-4} 
% \textbf{Head} & \textbf{\textit{Table column subhead}}& \textbf{\textit{Subhead}}& \textbf{\textit{Subhead}} \\
% \hline
% copy& More table copy$^{\mathrm{a}}$& &  \\
% \hline
% \multicolumn{4}{l}{$^{\mathrm{a}}$Sample of a Table footnote.}
% \end{tabular}
% \label{tab1}
% \end{center}
% \end{table}

% \begin{figure}[htbp]
% \centerline{\includegraphics{fig1.png}}
% \caption{Example of a figure caption.}
% \label{fig}
% \end{figure}

% Figure Labels: Use 8 point Times New Roman for Figure labels. Use words 
% rather than symbols or abbreviations when writing Figure axis labels to 
% avoid confusing the reader. As an example, write the quantity 
% ``Magnetization'', or ``Magnetization, M'', not just ``M''. If including 
% units in the label, present them within parentheses. Do not label axes only 
% with units. In the example, write ``Magnetization (A/m)'' or ``Magnetization 
% \{A[m(1)]\}'', not just ``A/m''. Do not label axes with a ratio of 
% quantities and units. For example, write ``Temperature (K)'', not 
% ``Temperature/K''.

% \section*{Acknowledgment}

% The preferred spelling of the word ``acknowledgment'' in America is without 
% an ``e'' after the ``g''. Avoid the stilted expression ``one of us (R. B. 
% G.) thanks $\ldots$''. Instead, try ``R. B. G. thanks$\ldots$''. Put sponsor 
% acknowledgments in the unnumbered footnote on the first page.

% \section*{References}
% Please number citations consecutively within brackets \cite{b1}. The 
% sentence punctuation follows the bracket \cite{b2}. Refer simply to the reference 
% number, as in \cite{b3}---do not use ``Ref. \cite{b3}'' or ``reference \cite{b3}'' except at 
% the beginning of a sentence: ``Reference \cite{b3} was the first $\ldots$''

% Number footnotes separately in superscripts. Place the actual footnote at 
% the bottom of the column in which it was cited. Do not put footnotes in the 
% abstract or reference list. Use letters for table footnotes.

% Unless there are six authors or more give all authors' names; do not use 
% ``et al.''. Papers that have not been published, even if they have been 
% submitted for publication, should be cited as ``unpublished'' \cite{b4}. Papers 
% that have been accepted for publication should be cited as ``in press'' \cite{b5}. 
% Capitalize only the first word in a paper title, except for proper nouns and 
% element symbols.

% For papers published in translation journals, please give the English 
% citation first, followed by the original foreign-language citation \cite{b6}.

% \begin{thebibliography}{00}
% \bibitem{b1} G. Eason, B. Noble, and I. N. Sneddon, ``On certain integrals of Lipschitz-Hankel type involving products of Bessel functions,'' Phil. Trans. Roy. Soc. London, vol. A247, pp. 529--551, April 1955.
% \bibitem{b2} J. Clerk Maxwell, A Treatise on Electricity and Magnetism, 3rd ed., vol. 2. Oxford: Clarendon, 1892, pp.68--73.
% \bibitem{b3} I. S. Jacobs and C. P. Bean, ``Fine particles, thin films and exchange anisotropy,'' in Magnetism, vol. III, G. T. Rado and H. Suhl, Eds. New York: Academic, 1963, pp. 271--350.
% \bibitem{b4} K. Elissa, ``Title of paper if known,'' unpublished.
% \bibitem{b5} R. Nicole, ``Title of paper with only first word capitalized,'' J. Name Stand. Abbrev., in press.
% \bibitem{b6} Y. Yorozu, M. Hirano, K. Oka, and Y. Tagawa, ``Electron spectroscopy studies on magneto-optical media and plastic substrate interface,'' IEEE Transl. J. Magn. Japan, vol. 2, pp. 740--741, August 1987 [Digests 9th Annual Conf. Magnetics Japan, p. 301, 1982].
% \bibitem{b7} M. Young, The Technical Writer's Handbook. Mill Valley, CA: University Science, 1989.
% \end{thebibliography}

% IEEE conference templates contain guidance text for composing and formatting conference papers. 
% Please ensure that all template text is removed from your conference paper prior to submission to 
% the conference. Failure to remove the template text from your paper may result in your paper not 
% being published.


